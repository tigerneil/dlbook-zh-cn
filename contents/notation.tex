\chapter{数学符号}
\label{ch:notation}

这一部分提供了一个简明的索引,描述整本书中使用的数学符号。如果你不熟悉任何对应的
数学概念,这个符号索引可能看起来很吓人。然而,不要绝望,我们在 2 --- 4 章里描述了
这些概念的大部分。

\begin{center}
\setstretch{1.5}
{\Large\bfseries 数和数组}\\
\vspace{1em}
\begin{tabular}{c l}
  $a$ & 标量(整数或实数)\\
  $\pmb{a}$ & 向量 \\
  $\pmb{A}$ & 矩阵 \\
  $\mathsf{A}$ & 张量 \\
  $\pmb{I}_n$ & $n$ 行 $n$ 列的单位矩阵 \\
  $\pmb{I}$ & 单位矩阵,其维度隐含在上下文中 \\
  $\pmb{e}^{(i)}$ & 标准的基向量 $[0, \ldots, 0, 1, 0, \ldots, 0]$,$1$ 的位置由 $i$ 确定 \\
  $diag(\pmb{a})$ & 方块对角矩阵,其对角线元素为 $\pmb{a}$ \\
  $\mathrm{a}$ & 标量随机变量 \\
  $\mathbf{a}$ & 向量值随机变量 \\
  $\mathbf{A}$ & 矩阵值随机变量
\end{tabular}
\end{center}

\vspace{1em}

\begin{center}
\setstretch{1.5}
{\Large\bfseries 集合和图}\\
\vspace{1em}
\begin{tabular}{c l}
  $\mathbb{A}$ & 集合 \\
  $\mathbb{R}$ & 实数集合 \\
  ${0,1}$ & 包含 $0$ 和 $1$ 的集合 \\
  ${0,1,\ldots,n}$ & $0$ 到 $n$ 的所有整数集合 \\
  $[a,b]$ & 包括 $a$ 和 $b$ 的实数区间 \\
  $(a,b]$ & 左开右闭区间,不包括 $a$ 但包括 $b$ \\
  $\mathbb{A}\backslash\mathbb{B}$ & 集合差,例如,包含有 $\mathbb{A}$ 的元素但其不在 $\mathbb{B}$ 中的集合 \\
  $\mathcal{G}$ & 图 \\
  $Pa_{\mathcal{G}}(x_i)$ & 图  $\mathcal{G}$ 中 $x_i$ 的父顶点
\end{tabular}
\end{center}

\vspace{1em}

\begin{center}
\setstretch{1.5}
{\Large\bfseries 索引}\\
\vspace{1em}
\begin{tabular}{c l}
  $a_i$ & 向量 $\pmb{a}$ 的第 $i$ 个元素,索引起始位置为 $1$ \\
  $a_{-i}$ & 除了第 $i$ 个元素的向量 $\pmb{a}$ 的所有元素 \\
  $A_{i,j}$ & 矩阵 $\pmb{A}$ 的 $i,j$ 位置元素
\end{tabular}
\end{center}
